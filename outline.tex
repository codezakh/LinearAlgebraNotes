\documentclass{article}
\usepackage{amssymb}
\usepackage{amsmath}
\usepackage{amsthm}


\title{Notes on Axler's Linear Algebra Done Right}
\author{Zaid Khan}
\date{\today}

\newtheorem{mytheorem}{Theorem}
\newtheorem{mydef}{Definition} %I will use this to start denoting definitions

\begin{document}

\maketitle

\clearpage


\tableofcontents


\clearpage


\section{Vector Spaces}
\subsection{R and C}
\subsection{Definition of a Vector Space}
\subsection{Subspaces}
A subset U of a vector space V is called a \textbf{subspace} if U is also a vector space, using the same addition and scalar multiplication defined on V.

\subsubsection{Conditions to a be a subspace}

\begin{itemize}

\item Additive Identity: \( 0 \in U \) .

\item Closed under addition.

\item Closed under multiplication.

\end{itemize}

\subsubsection{Sums of Subspaces}

Let \( U_{1} \ldots U_{m}     \) be subsets of V. The sum of \( U_{1} \ldots U_{m}     \) is the set of all possible sums of elements of \( U_{1} \ldots U_{m}     \) . 
\[U_{1} + \ldots + U_{m} = \left \{   u_{1} + u_{2} + \ldots + u_{m} : u_{1} \in U_{1} , u_{2} \in U _{2} \ldots , u_{m} \in U_{m}                      \right \} \]

The sum of subspaces is the smallest subspace containing all the summands.


\subsubsection{Direct Sums}


A sum \( U_{1} \ldots U_{m}     \) is called a direct sum if each element of \( U_{1} + \ldots + U_{m}     \) can be written in only one way. To simplify, there is only one way to write each element of the resulting space using a sum of elements of \( U_{1} \ldots U_{m}     \) .


\subsubsection{Conditions for a Direct Sum}


\begin{itemize}

\item Suppose that \( U_{1} \ldots U_{m}     \) are subspaces of V. Then \( U_{1 +} \ldots + U_{m}     \) is  a direct sum if and only if the only way to create the 0-vector is to take each \( u_{j} \) in the sum expression \( u_{1} + \ldots + u_{m} = 0   \) to be 0.


\item Corollary: Suppose that U and W are subspaces of V. Then \( U_{1} +\ldots + U_{m}     \) is a direct sum if and only if \( U \cap W = { 0 } \).

\end{itemize}





\section{Finite Dimensional Vector Spaces}

\subsection{Span and Linear Independence}
\subsubsection{Span}
The set of all linear combinations of a list of vectors \( v_{1}
\ldots v_{m} \) in \textit{V} is called the span of \( v_{1} \ldots v_{m} \).  The span of the empty list () is defined to be \(  \left \{ 0 \right \} \) .

\subsubsection{Finite Dimension}
A vector space is finite dimensional if there is a list of vectors that spans the space. Note that lists are, by definition, finite dimensional.


\subsubsection{Polynomials}

\begin{itemize}

\item A function \( p : \mathbb{F} \mapsto \mathbb{F} \) is called a polynomial with cefficients in F if there exists \(  a_{0} \ldots a_{m} \in \mathbb{F} \) such that \[   p(z) = a_{0} + a_{1} z + a_{2}z^{2} + \ldots + a_{m} z^{m} \quad \forall z \in \mathbb{F}   \] .

\item P(F) is the set of all polynomials in \textbf{F}.

\end{itemize}


\subsubsection{Linear Dependence}
\begin{itemize}

\item A list \(v_{1} \ldots v_{m}   \) is linearly independent if the only choice of \( a_{1} \ldots a_{m} \in \mathbb{F}  \) that makes \(  a_{1}v_{1} + \ldots + a_{m}v_{m} =0   \) is \(  a_{1} = \ldots = a_{m} = 0   \). That is to say, all the coefficients in the sum must be 0.


\item The empty list is declared to be linearly independent.

\end{itemize}


\subsection{Bases}

\subsubsection{Basis}
A basis of V is a list of vectors in V that is linearly independent and spans V.

\subsubsection{Criterion for basis}
A list \(v_{1} \ldots v_{n}  \) of vectors in \textit{V} is a basis of \textit{V} if and only if every \( v \in V\) can be written in the form \( v = a_{1}v_{1} + \ldots + a_{n}v_{n}  \) where \( a_{1} \ldots a_{n} \in \mathbb{F} \) and the list is linearly independent.

\subsubsection{Direct Sums, Subspaces, and Bases}
If V is finite dimensional and U is a subspace of V, there exists a subspace W of V such that \( V = W \oplus U \). To simply, every subspace of a finite dimensional vector space has a partner, which is also a subspace of V, which it forms a direct sum equal to V with.


\section{Linear Maps}
\subsection{The Vector Space of Linear Maps}

\subsubsection{Definition of Linear Maps}
A \textbf{linear map} from V to W is a function \(  T: V \mapsto W  \) such that the following properties are true:

\begin{itemize}

\item Additivity: \[ T(u+v) = T(u) + T(v) \]

\item Homogeneity: \[ T( \lambda v) = \lambda (Tv) \]

\end{itemize}

The set of all linear maps from V to W is denoted \(  \mathcal{L} (V,W)   \).

\subsubsection{Linear maps and basis of domain}

\begin{mytheorem} 
Suppose \( v_1 \ldots v_n \) is a basis of V and and \( w_1 \ldots w_n  \in W  \). Then \(  \exists T : V \mapsto W   \) such that \( Tv_j = w_j    \) for each j in \( 1 \ldots n \).

\end{mytheorem}

This theorem asserts that once we know the behavior of a linear map over the basis of vectors, the linear map is uniqely defined for all the vectors in the space.

\subsubsection{Algebraic Operations on L( V,W ) }

\begin{itemize}

\item Addition: \(  (S+T) (u) =  Su + Tu     \)

\item Scalar Multiplication: \(  ( \lambda T  ) (v)  = \lambda (Tv)    \)


\end{itemize}

\begin{mytheorem}
\(  \mathcal{L} (V,W)  \) is a vector space with the addition and scalar multiplication defined above.
\end{mytheorem}

\subsubsection{Product of Linear Maps}
The product of two linear maps is just function composition when the domains make sense.

\subsubsection{Algebraic Properties of Linear Maps}

\begin{itemize}

\item Associativity \(  (T_1 T_2 )  T_3  = T_1 (T_2 T_3)    \)

\item Identity \(     \)

\item Distributivity

\item Linear maps take 0 to 0.

\end{itemize}

Multiplication of linear maps is not commutative.

\subsection{Null Spaces and Ranges}


\subsubsection{Null Space}
\begin{itemize}

\item For $T \in L(V,W)$, the \textbf{null space} of $T$, denoted \textbf{null T} is the subset of $V$ containing those vectors that $T$ maps to 0. This can also be called the kernel.

\item Suppose $T \in L(V,W)$. Then null T is a subspace of V.

\end{itemize}

\subsubsection{Injectivity and Null Spaces}

\begin{itemize}
\item A function $T: V \mapsto W$ is called injective if Tu=TV implies u=v.
\item Let $T \in L(V,W)$. Then T is injective iff null T $={0}$.

\end{itemize}

\subsubsection{Definition of Range}

For T a function for V to W, the range of T is the subset of W consisting of those vectors which are of the form Tv for some $v \in V$.  This means the same as image.

\subsubsection{Ranges and Subspaces}
If $T \in L(V,W)$, then range T is a subspace of W. 

\subsubsection{Definition of Surjective}

A function $T: V \mapsto W$ is called surjective if its range equals W. This means the same as onto.

\subsubsection{Fundamental Theorem of Linear Maps}

Suppose V is finite dimensional and $T \in L(V,W)$. Then range T is finite-dimensional and dim V = dim null T + dim range T. 

\end{document}